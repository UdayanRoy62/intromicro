% Options for packages loaded elsewhere
\PassOptionsToPackage{unicode}{hyperref}
\PassOptionsToPackage{hyphens}{url}
%
\documentclass[
  letterpaper,
]{book}

\usepackage{amsmath,amssymb}
\usepackage{iftex}
\ifPDFTeX
  \usepackage[T1]{fontenc}
  \usepackage[utf8]{inputenc}
  \usepackage{textcomp} % provide euro and other symbols
\else % if luatex or xetex
  \usepackage{unicode-math}
  \defaultfontfeatures{Scale=MatchLowercase}
  \defaultfontfeatures[\rmfamily]{Ligatures=TeX,Scale=1}
\fi
\usepackage{lmodern}
\ifPDFTeX\else  
    % xetex/luatex font selection
\fi
% Use upquote if available, for straight quotes in verbatim environments
\IfFileExists{upquote.sty}{\usepackage{upquote}}{}
\IfFileExists{microtype.sty}{% use microtype if available
  \usepackage[]{microtype}
  \UseMicrotypeSet[protrusion]{basicmath} % disable protrusion for tt fonts
}{}
\makeatletter
\@ifundefined{KOMAClassName}{% if non-KOMA class
  \IfFileExists{parskip.sty}{%
    \usepackage{parskip}
  }{% else
    \setlength{\parindent}{0pt}
    \setlength{\parskip}{6pt plus 2pt minus 1pt}}
}{% if KOMA class
  \KOMAoptions{parskip=half}}
\makeatother
\usepackage{xcolor}
\setlength{\emergencystretch}{3em} % prevent overfull lines
\setcounter{secnumdepth}{5}
% Make \paragraph and \subparagraph free-standing
\ifx\paragraph\undefined\else
  \let\oldparagraph\paragraph
  \renewcommand{\paragraph}[1]{\oldparagraph{#1}\mbox{}}
\fi
\ifx\subparagraph\undefined\else
  \let\oldsubparagraph\subparagraph
  \renewcommand{\subparagraph}[1]{\oldsubparagraph{#1}\mbox{}}
\fi


\providecommand{\tightlist}{%
  \setlength{\itemsep}{0pt}\setlength{\parskip}{0pt}}\usepackage{longtable,booktabs,array}
\usepackage{calc} % for calculating minipage widths
% Correct order of tables after \paragraph or \subparagraph
\usepackage{etoolbox}
\makeatletter
\patchcmd\longtable{\par}{\if@noskipsec\mbox{}\fi\par}{}{}
\makeatother
% Allow footnotes in longtable head/foot
\IfFileExists{footnotehyper.sty}{\usepackage{footnotehyper}}{\usepackage{footnote}}
\makesavenoteenv{longtable}
\usepackage{graphicx}
\makeatletter
\def\maxwidth{\ifdim\Gin@nat@width>\linewidth\linewidth\else\Gin@nat@width\fi}
\def\maxheight{\ifdim\Gin@nat@height>\textheight\textheight\else\Gin@nat@height\fi}
\makeatother
% Scale images if necessary, so that they will not overflow the page
% margins by default, and it is still possible to overwrite the defaults
% using explicit options in \includegraphics[width, height, ...]{}
\setkeys{Gin}{width=\maxwidth,height=\maxheight,keepaspectratio}
% Set default figure placement to htbp
\makeatletter
\def\fps@figure{htbp}
\makeatother

\usepackage{makeidx}
\makeindex
\makeatletter
\@ifpackageloaded{bookmark}{}{\usepackage{bookmark}}
\makeatother
\makeatletter
\@ifpackageloaded{caption}{}{\usepackage{caption}}
\AtBeginDocument{%
\ifdefined\contentsname
  \renewcommand*\contentsname{Table of contents}
\else
  \newcommand\contentsname{Table of contents}
\fi
\ifdefined\listfigurename
  \renewcommand*\listfigurename{List of Figures}
\else
  \newcommand\listfigurename{List of Figures}
\fi
\ifdefined\listtablename
  \renewcommand*\listtablename{List of Tables}
\else
  \newcommand\listtablename{List of Tables}
\fi
\ifdefined\figurename
  \renewcommand*\figurename{Figure}
\else
  \newcommand\figurename{Figure}
\fi
\ifdefined\tablename
  \renewcommand*\tablename{Table}
\else
  \newcommand\tablename{Table}
\fi
}
\@ifpackageloaded{float}{}{\usepackage{float}}
\floatstyle{ruled}
\@ifundefined{c@chapter}{\newfloat{codelisting}{h}{lop}}{\newfloat{codelisting}{h}{lop}[chapter]}
\floatname{codelisting}{Listing}
\newcommand*\listoflistings{\listof{codelisting}{List of Listings}}
\makeatother
\makeatletter
\makeatother
\makeatletter
\@ifpackageloaded{caption}{}{\usepackage{caption}}
\@ifpackageloaded{subcaption}{}{\usepackage{subcaption}}
\makeatother
\ifLuaTeX
  \usepackage{selnolig}  % disable illegal ligatures
\fi
\usepackage{bookmark}

\IfFileExists{xurl.sty}{\usepackage{xurl}}{} % add URL line breaks if available
\urlstyle{same} % disable monospaced font for URLs
\hypersetup{
  pdftitle={Introduction to Microeconomics: Lecture Notes},
  pdfauthor={Udayan Roy},
  hidelinks,
  pdfcreator={LaTeX via pandoc}}

\title{Introduction to Microeconomics: Lecture Notes}
\author{Udayan Roy}
\date{2024-06-09}

\begin{document}
\frontmatter
\maketitle

\renewcommand*\contentsname{Table of contents}
{
\setcounter{tocdepth}{2}
\tableofcontents
}
\mainmatter
\bookmarksetup{startatroot}

\chapter*{Preface}\label{preface}
\addcontentsline{toc}{chapter}{Preface}

\markboth{Preface}{Preface}

These are my lecture notes for a course called \emph{Introduction to
Microeconomics} that I have taught for over three decades.

The required textbook for my course is \emph{Principles of
Microeconomics} by
\href{https://simple.wikipedia.org/wiki/N._Gregory_Mankiw}{N. Gregory
Mankiw}. Although Mankiw's book is excellent, I have constantly felt the
need to make small adjustments. Until recently, I had been making these
adjustments in the PowerPoint slides I use in the classroom. But the
availability of new technology -- especially
\href{https://www.r-project.org/}{\emph{R}},
\href{https://posit.co/download/rstudio-desktop/}{\emph{R Studio}},
\href{https://quarto.org/docs/books/}{\emph{quarto books}}, and
\href{https://github.com/}{\emph{Github}} -- has made it easy for me to
gather my slides and give them an online form that actually resembles a
textbook.

I would be happy to get feedback.

\bookmarksetup{startatroot}

\chapter{Introduction to Economics}\label{sec-introduction}

\section{Chapter Outline}\label{chapter-outline}

\begin{itemize}
\item
  What is economics?
\item
  What is the use of economics?
\item
  What are economists expected to do?
\item
  How do economists do what they are expected to do?
\item
  Why does the economist's method sometimes fail?
\item
  What is macroeconomics?
\item
  What is microeconomics?
\end{itemize}

\section{What is Economics?}\label{what-is-economics}

Economics is the study of our responses to scarcity, and the
consequences of those responses.

Scarcity -- in economics -- is the fact that we can't always get what we
want.

None of us has Aladdin's magic lamp.

\section{What is the Use of
Economics?}\label{what-is-the-use-of-economics}

Scarcity compels us to come up with less wasteful ways of running our
societies.

We all want progress, and we all want to reduce hunger, poverty and
inequality. But our resources are finite. Therefore, we can't afford to
run our societies in wasteful and inefficient ways. That's where good
economic policies have a crucial role to play. That's where the
economist can contribute.

The goals of a nation should not be set by economists. In democracies,
it is usually the job of the elected representatives of the people to
determine the nation's goals. Once the goals have been determined, it is
the job of the economist to outline the various policies by which the
nation's goals may be reached (assuming the goals are reachable).

The economist must also predict the likely benefits and costs of each of
the various policies by which the nation's goals may be achieved.

The people's representatives can then pick the policy they like best.

\subsection{Wasted Resources: Stuck in
Traffic}\label{wasted-resources-stuck-in-traffic}

We waste a lot of time stuck in traffic. Economists would want to find a
way to reduce this waste of time. Building more roads may not always be
possible, may not solve the problem, and may be costly in any case.
Charging car owners for the use of a road may be the way to go.
Investing in or subsidizing public transport is another option.

\subsubsection{Wasted Resources: Stuck in Traffic: News
Item}\label{wasted-resources-stuck-in-traffic-news-item}

\href{https://www.wsj.com/articles/charging-drivers-for-road-use-is-popular-with-economists-less-so-with-drivers-11630245780}{Charging
Drivers for Road Use Is Popular With Economists, Less So With Drivers}
By David Harrison, The Wall Street Journal, Aug.~29, 2021

\subsection{Wasted Resources:
Unemployment}\label{wasted-resources-unemployment}

What can we do to reduce this waste of resources? Cut taxes to encourage
people to go shopping? Have the government spend more roads and bridges?
Make overtime work illegal? Limit imports?

\subsection{Wasted Resources: Environment
Destruction}\label{wasted-resources-environment-destruction}

How can we ensure that the resources that sustain life on this planet
are conserved?

\subsection{Addressing Ethical Imperatives: Health
Care}\label{addressing-ethical-imperatives-health-care}

How can we ensure that people do not need to worry about something as
basic as healthcare?

\subsection{Addressing Ethical Imperatives:
Inequality}\label{addressing-ethical-imperatives-inequality}

How can we ensure that incomes are more equally shared?

This assumes that a more equal nation is what the people want. (Remember
that it is not the economist's job to say what the nation should or
should not want.) Raise taxes on the rich? Invest in and subsidize
higher education? Change existing laws to strengthen labor unions?

\subsection{Addressing Ethical Imperatives: Intergenerational
Mobility}\label{addressing-ethical-imperatives-intergenerational-mobility}

How can we keep the American Dream---the idea that each generation lives
better than their parents---alive? Universal Basic Income? Baby Bonds?
Require politicians to raise election campaign money through small
donations? Tuition-free college?

\section{What Are Economists Expected to
Do?}\label{what-are-economists-expected-to-do}

As we have just seen, when asked specific policy-related questions, the
economist has to think hard and:

\begin{itemize}
\item
  Identify the list of the options available to society, and
\item
  Make predictions of the consequences---costs and benefits---that would
  follow from each of those options.
\end{itemize}

The democratic process must then decide which option to pursue.

\subsection{Government policies can make a difference: in ways good and
bad:
Examples}\label{government-policies-can-make-a-difference-in-ways-good-and-bad-examples}

Government's Pandemic Response Turned a Would-Be Poverty Surge Into a
Record Poverty Decline, By Danilo Trisi, Center for Budget and Policy
Priorities, August 29, 2023

\subsubsection{When Economic Policies Go Wrong: Joseph
Stalin}\label{when-economic-policies-go-wrong-joseph-stalin}

Clearly, the stakes are very high in getting our economic policies
right. The Soviet leader Stalin is believed to have caused about 40
million deaths during the 1930s in the then Soviet Union in an attempt
to collectivize agriculture. For more on Stalin's policies and their
effects see:

\begin{itemize}
\item
  \emph{Harvest of Sorrow: Soviet Collectivization and the
  Terror-Famine} by Robert Conquest, Oxford University Press, New York,
  NY, 1986, ISBN 0195051807.
\item
  \emph{Bloodlands: Europe Between Hitler and Stalin} by Timothy Snyder,
  Basic Books, New York, NY, 2010, ISBN 978-0465002399.
\end{itemize}

\subsubsection{When Economic Policies Go Wrong---Mao
Zedong}\label{when-economic-policies-go-wrongmao-zedong}

During 1958-61 there was a famine in China that is estimated to have
killed 30 million people. Those deaths were largely due to the Chinese
leader Mao Zedong's failed economic policy---grandly called ``the great
leap forward''---of the forced industrialization of China's agricultural
economy.

For more on Mao's policies and the famine that they caused please read:

\begin{itemize}
\item
  Tombstone: The Great Chinese Famine, 1958-1962 by Yang Jisheng,
  Farrar, Straus and Giroux, New York, NY, 2012, ISBN 978-0374277932.
\item
  Hungry Ghosts: Mao's Secret Famine by Jasper Becker, Free Press, New
  York, NY, 1996, ISBN: 068483457X.
\item
  Mao's Great Famine: The History of China's Most Devastating
  Catastrophe by Frank Dikotter, Walker \& Co., New York, NY, 2010,
  ISBN: 978-0-8027-7768-3.
\end{itemize}

\section{How Do Economists Do What They Are Expected to
Do?}\label{how-do-economists-do-what-they-are-expected-to-do}

\subsection{Economists make simplifying
assumptions}\label{economists-make-simplifying-assumptions}

The simplifications help economists make predictions about the likely
consequences of different policy options.

Economists usually disagree about their predictions. This is because
different economists make different simplifying assumptions.

They try to use data to sort out their disagreements. If all goes well,
economists may come up with unambiguous and useful advice for policy
makers.

An actual economy is extremely complex. So, it is hard to think about
how it would respond to, say, an increase in income tax rates.
Therefore, it would be hard to make even a theoretical prediction of a
tax hike's effect on, say, the unemployment rate.\footnote{Remember,
  this is the sort of prediction that the nation might need.} So
economists need to make simplifying assumptions in their analyses. They
need to imagine a simpler economy because it might be easier to make a
prediction for the simplified imaginary economy than for the complex
actual economy.

These simplifying assumptions must be very carefully chosen, however, so
that the hypothetical simplified economy is not too dissimilar to the
actual economy, and is, at the same time, easier to analyze than the
actual economy.

\subsubsection{Predictions: Examples}\label{predictions-examples}

\begin{itemize}
\item
  If income tax rates are increased, the nation's unemployment rate will
  increase.
\item
  If Florida has a severe winter, the price of orange juice will
  increase.
\item
  If the price of imported oil goes up, the nation's gross domestic
  product will decrease.
\item
  If the tax per airline ticket is increased, the price of hotel rooms
  will decrease.
\end{itemize}

\subsubsection{Predictions: Disagreements Are
Inevitable}\label{predictions-disagreements-are-inevitable}

There is no universally accepted way for economists to decide which
simplifying assumptions are the most appropriate. Different economists
when asked the same question---say, How will an income tax hike affect
unemployment?---may make different simplifying assumptions in their
analyses. Therefore, they may end up making different predictions.

\subsubsection{Data Helps to Sort Out
Disagreements}\label{data-helps-to-sort-out-disagreements}

When economists disagree, the right kind of historical evidence may help
them decide which economist's prediction to trust. A large part of the
economist's job is to dig up evidence from the past, and use the
evidence to test the clashing theories that various economists may
propose.

\section{Why Does the Economist's Method Sometimes
Fail?}\label{why-does-the-economists-method-sometimes-fail}

We have just seen that different economists may give different answers
(predictions) for the same question. Those disagreements can't always be
sorted out. (Why?) This leaves the general public puzzled and annoyed.

\subsection{Economists' Disagreements Can't Always Be Sorted
Out}\label{economists-disagreements-cant-always-be-sorted-out}

Often there isn't enough data. When there isn't enough data, one may not
be able to choose between clashing theories or predictions.

Economists generally can't do experiments. Even if there is a lot of
historical data, there is no guarantee that a study of the past will
help in identifying the best theory or prediction. If you toss a coin
ten thousand times, you would be no better at predicting the ten
thousand and first toss as you were at predicting the first toss.

\section{What Is Macroeconomics? What Is
Microeconomics?}\label{what-is-macroeconomics-what-is-microeconomics}

Macroeconomics deals with questions about variables that describe the
economy of an entire nation. Microeconomics deals with questions related
to individual economic agents, such as households and firms.

\subsection{Macroeconomics}\label{macroeconomics}

Macroeconomics deals with issues related to data that give summary
descriptions of the economy of an entire nation. A macroeconomist would
ponder questions such as: What would happen to Uzbekistan's unemployment
rate if Japan suddenly stops trading with Uzbekistan and what policy
should the government of Uzbekistan then follow? The focus would always
be on Uzbekistan as a whole.

\subsection{Microeconomics}\label{microeconomics}

Microeconomics deals with questions related to economic variables that
describe a sub-national entity, typically individual economic agents,
such as households and firms.

\subsection{Macro and Micro Are
Related}\label{macro-and-micro-are-related}

One cannot really do macroeconomics without simultaneously doing
microeconomics. One cannot analyze an economy without studying the
behavior of the individual economic units that make up that economy.

Conversely, the decisions by individuals are often guided by their
expectations about incomes, interest rates, inflation, and the like. And
these expectations cannot be understood without an analysis of the
economy as a whole.

However, in macroeconomics the microeconomic underpinnings are
de-emphasized. Conversely, in microeconomics the macroeconomic
foundations of people's expectations are de-emphasized.

This course focuses on microeconomics.

\bookmarksetup{startatroot}

\chapter{The Gains from Trade}\label{sec-trade}

\section{Chapter Outline}\label{chapter-outline-1}

\begin{itemize}
\item
  Why study trade?
\item
  Why Do We Trade?

  \begin{itemize}
  \item
    Because our preferences are different
  \item
    Because we are differently endowed with skills, technologies, and
    natural resources

    \begin{itemize}
    \item
      Opportunity costs and trade
    \item
      Comparative advantage and the gains from trade
    \item
      Opportunity costs and technology
    \item
      Graphing production possibilities
    \end{itemize}
  \item
    Because there are efficiency advantages to doing one thing rather
    than many things
  \end{itemize}
\end{itemize}

\section{Why Should We Study Trade?}\label{why-should-we-study-trade}

People trade with each other---a lot. Do you know anyone who makes all
the things he or she consumes? To understand our world we need to
understand why people trade so much. We need to understand whether trade
is good for us or bad for us. Understanding this is important precisely
because we trade a lot.

\section{Why Do We Trade?}\label{why-do-we-trade}

\begin{itemize}
\item
  Because our preferences are different.
\item
  Because we are differently endowed with skills, technologies, and
  natural resources.
\item
  Because there are efficiency advantages to doing one thing rather than
  many things.
\end{itemize}

\subsection{Why Do We Trade? Because Our Preferences are
Different}\label{why-do-we-trade-because-our-preferences-are-different}

Imagine a two-country world. Fishing is the only work people do. When
people go out to catch fish, equal amounts of salmon and cod always get
caught in the nets. The two countries are identical except for different
preferences: People in one country like salmon and people in the other
country like cod.

Naturally, this difference in preferences will lead to trade: the salmon
lovers will export the cod they catch to the cod lovers, and vice versa.

But, this kind of trade is not very relevant to the world we live in.
So, I'll move on.

\subsection{Why Do We Trade? Because we are differently endowed with
skills, technologies, and natural
resources}\label{why-do-we-trade-because-we-are-differently-endowed-with-skills-technologies-and-natural-resources}

\subsubsection{Imagine a simplified version of our
world}\label{imagine-a-simplified-version-of-our-world}

Imagine a world with:

\begin{itemize}
\item
  only two goods: potatoes and meat, and
\item
  only two people: a potato farmer and a cattle rancher.
\end{itemize}

What amounts of each good should each of them produce? Should they
trade?

\paragraph{Quick Detour}\label{quick-detour}

Why am I assuming a world with only two goods and only two people? We
have seen in Chapter~\ref{sec-introduction} that simplicity is often key
to clarifying an idea.But in that case why not assume a world with just
one good and/or just one person?

\subsubsection{Make or Buy?}\label{make-or-buy}

If you want something, should you make it yourself? Or should you make
something else and then trade it for the thing you need?

To understand whether a person would make a particular thing or buy it
from another person, we need to compare, for that person, the cost of
making it with the price of buying it (from the other person). So, let
us begin with the cost of making a thing.

\subsubsection{Opportunity Cost}\label{opportunity-cost}

In our story, the farmer can produce both meat and potatoes. However, as
the farmer has a finite amount of the resources needed for production, a
one-unit increase in his meat production will cause a decrease in his
potato production The decrease in the farmer's potato production that is
caused by a one-unit increase in his meat production is his opportunity
cost of meat.

Note that the opportunity cost is not measured in dollars. The
opportunity cost of additional meat production is measured by the amount
of potato production that is sacrificed.

\subsubsection{Opportunity Cost:
Generalized}\label{opportunity-cost-generalized}

Stepping away briefly from our meat-potatoes example \ldots{} The
opportunity cost of obtaining some thing is everything that you'll have
to give up to get that thing.

Can you apply the concept of opportunity cost to your own life? What is
the opportunity cost, for you, of taking an additional math course? List
all the activities you normally engage in every day. Think about all the
sacrifices you will have to make if you were to enroll in an additional
math course. That's your opportunity cost of taking an additional math
course this semester.

\subsubsection{Opportunity Cost and Trade: What if we were all pretty
similar?}\label{opportunity-cost-and-trade-what-if-we-were-all-pretty-similar}

Suppose the opportunity cost of an ounce of meat is 3 ounces of potatoes
for both Farmer and Rancher. Will they trade?

No.~Trade would be pointless in this case.

One can easily imagine our Farmer making the following piych to our
Rancher: ``For me, the cost of making 1 ounce of meat is 3 ounces of
potatoes. I'll buy 1 ounce of meat from you if you charge a price that
is less than 3 ounces of potatoes. Deal?''

This is probably how our Rancher will respond: ``For me, the cost of
making 1 ounce of meat is 3 ounces of potatoes. So, I can sell you 1
ounce of meat only for a price that is more than 3 ounces of potatoes.
So, sorry, no deal!''

This exchange illustrates the idea that trade is pointless when
opportunity costs are the same for all producers.

\textbf{Key idea:} If people have similar opportunity costs for some
commodity, then they would probably not trade in that commodity with
each other.

\subsubsection{Opportunity Cost and Trade: What if we were all pretty
different?}\label{opportunity-cost-and-trade-what-if-we-were-all-pretty-different}

Let's change our story a bit. Suppose the opportunity cost of an ounce
of meat is 4 ounces of potatoes for Farmer and 2 ounces of potatoes for
Rancher. Now, will they trade?

Yes! Rancher will offer to sell meat to farmer at a price between 2 and
4 ounces of potatoes per ounce of meat. Farmer will gladly accept. Both
farmer and rancher will be better off.

One can easily imagine our Farmer making the following piych to our
Rancher: ``For me, the cost of making 1 ounce of meat is 4 ounces of
potatoes. I'll buy 1 ounce of meat from you if you charge a price that
is less than 4 ounces of potatoes. Deal?''

And this is probably how our Rancher will respond: ``For me, the cost of
making 1 ounce of meat is 2 ounces of potatoes. So, I can sell you 1
ounce of meat only for a price that is more than 2 ounces of potatoes.
Deal!''

Rancher will offer to sell meat to farmer at a price between 2 and 4
ounces of potatoes per ounce of meat. Farmer will gladly accept. Rancher
will increase his meat production \ldots{} and, therefore, decrease his
potato production. Rancher will sell meat to Farmer and get paid in
potatoes.

Conversely, Rancher will offer to sell meat to Farmer at a price between
2 and 4 ounces of potatoes per ounce of meat. Farmer will gladly accept.
Farmer will increase his potato production \ldots{} and, therefore,
decrease his meat production. Farmer will sell potatoes to Rancher and
get paid in meat.

\subsubsection{Opportunity Cost and Trade: Key
Ideas}\label{opportunity-cost-and-trade-key-ideas}

Our example suggests the following important ideas that connect
opportunity costs and trade.

\begin{itemize}
\item
  If people have different opportunity costs for some commodity, then
  they will trade in that commodity with each other.
\item
  For any commodity, the person whose opportunity cost is lower will be
  the seller and the person whose opportunity cost is higher will be the
  buyer.
\item
  The price at which the trading occurs will be somewhere between the
  two traders' opportunity costs.
\item
  When trade becomes possible, every producer increases his production
  of the commodity for which his opportunity cost is lower \ldots{} and
  decreases his production of the commodity for which his opportunity
  cost is higher.
\item
  Trade causes people to do more of what they do well and less of what
  they don't do well. This is the key reason why we trade so much.
\end{itemize}

\subsubsection{Comparative Advantage}\label{comparative-advantage}

Key Definition: The producer with the lower opportunity cost in the
production of a commodity is said to have a comparative advantage in the
production of that commodity.

In our example, in potato production, Farmer has the comparative
advantage. In meat production, Rancher has the comparative advantage.

Trade makes people specialize in the production of the good they have a
comparative advantage in. In our example, Rancher has a comparative
advantage in producing meat. Trade gives the rancher the incentive to
expand meat production for sale (export) to the farmer. That is, trade
gives the rancher the incentive to specialize in what he does best.

\subsection{Comparative advantage and the gains from
trade}\label{comparative-advantage-and-the-gains-from-trade}

Why Is Trade Good for Us? In our example, trade benefits both the Farmer
and the Rancher by enabling each person to do only what he is better
suited to do. Imagine what it would be like if you were required to
produce everything that you needed. The situation would be similarly
awful for a country that either chose not to trade with other countries
or was forced to end all trade with other countries.

\subsubsection{Theory of Comparative
Advantage}\label{theory-of-comparative-advantage}

The Theory of Comparative Advantage says that if each person specializes
in producing what he or she has a comparative advantage in, then total
production of every good can increase. As a result, trade can benefit
everybody.

In our example, the theory says that if Farmer specializes in potatoes
and Rancher specializes in meat, the total production of meat can be
increased and the total production of potatoes can also be increased. As
a result, if Rancher and Farmer then trade, they could both benefit. But
is this theory true? Theory of Comparative Advantage---Proof Suppose
Farmer increases his production of potatoes by 4 ounces. Then, according
to Table 1, his production of meat must decrease by 1 ounce. Suppose
Rancher increases his production of meat by 1.5 ounces. Then his
production of potatoes must decrease by 3 ounces. Therefore, by making
these two people specialize according to their comparative advantages,
it is possible to increase the total output of meat by 0.5 ounces and of
potatoes by 1 ounce. Wow! We have just witnessed a miracle---the miracle
of trade. For an individual, it is impossible to make more of one good
without making less of some other good. But for the world as a whole, it
is possible to produce more of all goods simultaneously if we embrace
trade. The Legacy of Adam Smith and David Ricardo Adam Smith In his 1776
book An Inquiry into the Nature and Causes of the Wealth of Nations,
Adam Smith performed a detailed analysis of trade and economic
interdependence, which economists still adhere to today. David Ricardo
In his 1816 book Principles of Political Economy and Taxation, David
Ricardo developed the principle of comparative advantage as we know it
today. Opportunity costs are related to technology Opportunity Costs and
Trade We have just seen that opportunity costs are crucial for
understanding trade What makes opportunity costs vary from person to
person or from country to country? One answer is technology Technology
Explains Opportunity Cost Key idea: Different people/countries may have
different technologies and this causes them to have different
opportunity costs

Let us now see a numerical example of how differences in technology lead
to differences in opportunity costs

Production Technologies of the Farmer and Rancher Opportunity Costs of
Farmer 1 ounce of meat → 60 minutes. 1 ounce of potatoes → 15 minutes. 4
ounces of potatoes → 60 minutes. Therefore, Farmer's opportunity cost of
1 ounce of meat is 4 ounces of potatoes. Opportunity Costs of Farmer 1
ounce of potatoes → 15 minutes. 1 ounce of meat → 60 minutes. ¼ ounce of
meat → 15 minutes. Therefore, Farmer's opportunity cost of 1 ounce of
potatoes is ¼ ounces of meat.

Opportunity Costs of Rancher 1 ounce of meat → 20 minutes. 1 ounce of
potatoes → 10 minutes. 2 ounces of potatoes → 20 minutes. Therefore,
Rancher's opportunity cost of 1 ounce of meat is 2 ounces of potatoes.

Opportunity Costs of Rancher 1 ounce of potatoes → 10 minutes. 1 ounce
of meat → 20 minutes. ½ ounce of meat → 10 minutes. Therefore, Rancher
's opportunity cost of 1 ounce of potatoes is ½ ounces of meat.

Reminder: Opportunity Costs and Comparative Advantage Farmer has a
comparative advantage in potatoes and Rancher has a comparative
advantage in meat. Technological differences are an important reason why
we trade To sum up, we have so far seen that Trade happens if and only
if opportunity costs vary from person to person (or from country to
country) Differences in technological abilities can lead to differences
in opportunity costs If you are curious: Absolute Advantage and
Comparative Advantage If Farmer can make an ounce of potatoes in less
time than Rancher needs to do the same, then Farmer is said to have an
absolute advantage in making potatoes On the other hand, as we have seen
already, if Farmer can make an ounce of potatoes at a lower opportunity
cost than Rancher can, then Farmer is said to have a comparative
advantage in making potatoes If you are curious: Absolute Advantage and
Comparative Advantage At one point, economists thought that two people
would trade if and only if each had an absolute advantage in the
production of some commodity. David Ricardo, a nineteenth-century
British economist, later showed that absolute advantage is irrelevant.
Two people would trade if and only if each had a comparative advantage
in the production of some commodity.

Gains from Trade have Nothing to Do with Technological Superiority If
you are curious: Absolute Advantage and Comparative Advantage If you are
curious: Absolute Advantage and Comparative Advantage If you are
curious: Absolute Advantage and Comparative Advantage Exercise:
calculation of opportunity costs from technology We have seen how
opportunity costs can be calculated from the 2nd and 3rd columns (blue
border) of the technology table below But can you do it using the 4th
and 5th columns (brown border) instead?

\subsection{Graphing production
possibilities}\label{graphing-production-possibilities}

Rancher's Production Possibilities: Further Details Rancher's Production
Possibilities Frontier Rancher's Production Possibilities Frontier The
Production Possibilities Frontier The PPF graphically illustrates the
fact that we face trade-offs If either person increases his production
of meat, his production of potatoes must decrease. When there is no
trade, each person must consume what he produces. In that case, if
either person increases his consumption of meat, his consumption of
potatoes must decrease. Farmer's Production Possibilities The Farmer's
Production Possibilities Frontier The Production Possibilities Frontier
Can Shift More Meat and More Potatoes? It may be possible to increase
one's consumption of both meat and potatoes---as in the last
slide---if\ldots{} More resources or better resources become available,
or Technology becomes more advanced, or Farmer and Rancher begin to
trade More Meat and More Potatoes? Trade can increase the overall
production---and consumption---of both goods even if resources and
technology remain unchanged. This is the miracle of trade. The
Production Possibilities Frontier The production possibilities frontier
is a graph that shows the combinations of output that the economy can
produce, given the available factors (resources) of production and the
available production technology. The Production Possibilities Frontier
The Production Possibilities Frontier Concepts illustrated by the
production possibilities frontier Efficiency Trade-offs Opportunity cost
Economic growth A Shift in the Production Possibilities Frontier Q: Why
Do We Trade? A:

\section{Beyond Comparative Advantage: Other Reasons Why Trade
Occurs}\label{beyond-comparative-advantage-other-reasons-why-trade-occurs}

There are efficiency advantages to doing one thing rather than many
things. Differences in Opportunity Costs Can't be the Only Explanation
for Trade. Why is Canada our main trade partner despite being so similar
to the US?

Trade allows us to fully utilize the benefits of bulk production by
allowing each country's production to be sold everywhere. Trade
intensifies competition and squeezes out inefficient production.

\section{What next?}\label{what-next}

We have seen why trade occurs? But when trade does occur, at what price
will it occur? In showing how trade can make Farmer and Rancher better
off, I worked out an example of how trade could occur. Specifically, I
showed that if 1.25 ounces of meat are traded for 3.5 ounces of
potatoes, both Farmer and Rancher would be better off. But will trade
take place? And if it does, at what price will people trade? That's the
subject of the next chapter.

\bookmarksetup{startatroot}

\chapter{The Market Forces of Demand and
Supply}\label{sec-supply_demand}

In Chapter~\ref{sec-trade}, I discussed some of the reasons why people
trade with each other. Unfortunately, that chapter had little to say
about the price at which one good is traded for another. In this
chapter, I will begin to fill that gap.

I will assume an economy in which people trade a great deal. I will
assume that there are well developed \emph{markets} in which each traded
commodity -- a good or a service or an asset -- is bought and sold.

A good starting point in the description of all the buying and selling
that goes on in the market for a traded commodity are two variables:
\emph{price} and \emph{quantity}. If I am interested in the market for,
say, apples, I would want to know the price of an apple and the quantity
of apples bought and sold, say, every month.

I want a \emph{theory} of the price and the quantity. That is, I want a
consistent and systematic way to think about observed changes in the
price and the quantity.

\section{Why do we need a theory of prices and
quantities?}\label{why-do-we-need-a-theory-of-prices-and-quantities}

Recall from Chapter~\ref{sec-introduction} that in economics we need to
be able to \emph{predict} the economic consequences of:

\begin{itemize}
\tightlist
\item
  alternative policies, and
\item
  events outside our control that we need to be prepared for.
\end{itemize}

For both individuals and for societies, reliable predictions guide our
choices. The reliability of a prediction cannot be established without
careful analysis of data (obtained either from historical observation or
from experiments). But in many cases we do not have data-based
measurements of the reliability of a prediction. In such cases, we can
at least check whether a prediction is based on logically consistent
reasoning.

A prediction needs to come out of a theory, which is a systematic and
rational use of logic. In other words, the mental tool we use to make
predictions is a theory.\footnote{A theory is of no use if its
  predictions are inaccurate. For any theory, we need to measure how
  accurate the predictions of the theory are. We may be able to do this
  using the tools of
  \href{https://en.wikipedia.org/wiki/Statistics}{statistics} and
  \href{https://en.wikipedia.org/wiki/Econometrics}{econometrics}. Those
  subjects are not discussed in these lectures.}

To further explore the need for a theory of prices, consider yourself
casting a vote in an election. You may want to make a rational
comparison of the policies espoused by candidates \emph{A}, \emph{B},
and \emph{C}. You will need to imagine the state of your community under
policies \emph{A}, \emph{B}, and \emph{C}. To do that you will need to
imagine the behavior of the people of your community under policies
\emph{A}, \emph{B}, and \emph{C}. To do that you will need to figure out
the incentives that the people of your community will face under
policies \emph{A}, \emph{B}, and \emph{C}. And, because prices are often
(if not always) a crucial incentive (or even \emph{the} crucial
incentive), you will need to figure out the prices that the people of
your community will face under policies \emph{A}, \emph{B}, and
\emph{C}. And to have a logically consistent way to predict the prices
that the people of your community will face under policies \emph{A},
\emph{B}, and \emph{C}, you will need \emph{a theory of prices}.

This is why the theory of prices is so crucial. There is hardly a single
social issue where a rational comparison of policy options would not
require a theory of how prices would be affected by the policy options
being compared.

And that is why this entire course is about prices. Indeed, another name
for microeconomics is \emph{price theory}.

\section{The Punchline}\label{the-punchline}

Before we begin this course on prices, let's give away the punchline: In
a market-based economy, \emph{the price of a commodity will be high if}
the commodity is:

\begin{itemize}
\tightlist
\item
  Highly desired,
\item
  Costly to produce, and/or
\item
  Sold under little or no competition among sellers.
\end{itemize}

The rest of this course is an extended discussion of the previous
sentence. I begin, in this chapter, with a simple yet fundamental
theory: the theory of demand and supply.

\section{The theory of demand and
supply}\label{the-theory-of-demand-and-supply}

The theory of demand and supply is a simple example of an economic
theory. It can be used to make predictions about:

\begin{itemize}
\tightlist
\item
  the price and
\item
  the quantity traded, of some traded commodity.
\end{itemize}

For example, the theory of demand and supply can help you make a
prediction about the effect of unusually cold weather on the price and
the quantity traded of home heating oil in New York.

Keep in mind that it is crucial to be able to predict prices because
prices are the most important economic incentives in a market-based
economy.

\subsection{Assumptions: Perfect
Competition}\label{assumptions-perfect-competition}

The theory of supply and demand assumes that each commodity is traded by
buyers and sellers dealing directly with each other in the market for
that commodity. Coffee is bought and sold in the market for coffee,
apples are bought and sold in the market for apples, etc.\footnote{When
  I refer to the ``market for coffee'', I simply mean the buyers and
  sellers of coffee and the rules they obey when they trade.} Any buyer
in a market may buy from any seller and any seller may sell to any
buyer.

The theory of supply and demand also assumes that commodities are traded
in perfectly competitive markets. A \emph{perfectly competitive market}
is one in which:

\begin{itemize}
\tightlist
\item
  there are many buyers,
\item
  there are many sellers, and
\item
  all sellers sell the exact same product.
\end{itemize}

As a result, under perfect competition, the price of a commodity will be
the same in every single trade of that commodity. As there are many
buyers, no seller has any need to offer a special low price to a
particular buyer. Similarly, as there are may sellers, no buyer has any
need to pay a special high price to a particular seller. So, all buying
and selling must occur at the same price and we refer to that price as
\emph{the market price}.\footnote{The market price may, of course, be
  different in different situations. But all trade in a given situation
  must take place at the same price, which we call the market price.
  Under \href{https://en.wikipedia.org/wiki/Monopoly}{monopoly} -- a
  market with only one seller -- it is possible that the monopolist may
  charge different buyers different prices, a phenomenon called
  \href{https://en.wikipedia.org/wiki/Price_discrimination}{price
  discrimination}.}

As there are many buyers and many sellers all selling the same
commodity, each buyer and each seller has a negligible impact on the
market price: everybody is a \emph{price taker}.

\subsubsection{Exercise}\label{exercise}

Are the markets for these commodities perfectly competitive?

\begin{itemize}
\tightlist
\item
  Wheat
\item
  White cotton T-shirts
\item
  Automobiles
\item
  Cable TV in your locality
\item
  Electricity for home use in your locality
\end{itemize}

\subsubsection{Other kinds of markets}\label{other-kinds-of-markets}

What's the opposite of perfect competition? Imperfect competition, of
course. Specifically, we will discuss:

\begin{itemize}
\tightlist
\item
  Monopoly (one seller)
\item
  Monopolistic Competition (many sellers, differentiated products)
\item
  Oligopoly (small number of sellers)
\end{itemize}

\section{Demand}\label{demand}

How should we describe the behavior of buyers?

\subsection{Quantity demanded}\label{quantity-demanded}

Quantity demanded is the amount of a commodity that buyers are willing
and able to purchase.

The quantity demanded of a good/service depends on:

\begin{itemize}
\tightlist
\item
  The price of the good/service
\item
  The prices of related goods/services
\item
  Buyers' incomes
\item
  Buyers' tastes
\item
  Buyers' expectations about future prices and incomes
\item
  Number of buyers,
\item
  etc.
\end{itemize}

\subsection{Demand}\label{demand-1}

Demand is a full description of how the quantity demanded changes as the
price of the commodity changes.

\subsection{An Individual's Demand Schedule and Demand
Curve}\label{an-individuals-demand-schedule-and-demand-curve}

\subsection{Market Demand is the Sum of Individual
Demands.}\label{market-demand-is-the-sum-of-individual-demands.}

\subsection{Law of Demand}\label{law-of-demand}

The law of demand states that the quantity demanded of a good falls when
the price of the good rises, and vice versa, provided all other factors
that affect buyers' decisions are unchanged.

\subsection{Why Might Demand Increase?}\label{why-might-demand-increase}

How can we explain the difference in Catherine's behavior in situations
A and B? Why does she consume more in situation B at every possible
price?

\subsubsection{Shifts in the Demand Curve Caused by Changes In Consumer
Income}\label{shifts-in-the-demand-curve-caused-by-changes-in-consumer-income}

As income increases the demand for a \emph{normal good} will increase.

As income increases the demand for an \emph{inferior good} will
decrease.

Example: If restaurant food is a normal good, then Demand shifts right
when incomes rise. If fast food is an inferior good, then Demand shifts
left when incomes rise.

\subsubsection{Shifts in the Demand Curve Caused by Changes in Prices of
Related
Goods}\label{shifts-in-the-demand-curve-caused-by-changes-in-prices-of-related-goods}

When a \emph{decrease} in the price of one good \emph{decreases} the
demand for another good, the two goods are called \emph{substitutes}.

When a \emph{decrease} in the price of one good \emph{increases} the
demand for another good, the two goods are called \emph{complements}.

Example: If Pepsi and Coke are substitutes, then Pepsi's demand
decreases (shifts left) when Coke's price decreases. If Cars and
gasoline are complements, then the demand for cars increases (shifts
right) when the price of gas decreases.

\subsection{The Law of Demand:
Explanations}\label{the-law-of-demand-explanations}

There are two ways to explain the Law of Demand:

\begin{itemize}
\tightlist
\item
  Substitution effect
\item
  Income effect
\end{itemize}

\subsubsection{The Law of Demand: Explanations: Substitution
Effect}\label{the-law-of-demand-explanations-substitution-effect}

When the price of a good decreases, consumers substitute that good
instead of other competing (substitute) goods.

\subsubsection{The Law of Demand: Explanations: Income
Effect}\label{the-law-of-demand-explanations-income-effect}

A decrease in the price of a commodity is essentially equivalent to an
increase in consumers' income. That is, Lower Prices = Higher Income.
Consumers respond to a decrease in the price of a commodity as they
would to an increase in income. They increase their consumption of a
wide range of goods, including the good that had a price decrease.

We have used the substitution effect and the income effect to show that
the Law of Demand is true for normal goods.

Can you imagine an example where the Law of Demand is not true?

\section{Supply}\label{supply}

How can we describe the behavior of sellers?

\subsection{Quantity supplied}\label{quantity-supplied}

Quantity supplied is the amount of a good that sellers are willing and
able to sell.

The quantity supplied of a good/service depends on:

\begin{itemize}
\tightlist
\item
  The price of the good/service
\item
  The prices of raw materials and labor
\item
  Technology
\item
  Number of sellers,
\item
  etc.
\end{itemize}

\subsection{Supply}\label{supply-1}

Supply is a full description of how the quantity supplied of a commodity
responds to changes in its price.

\subsubsection{Individual's supply schedule and supply
curve}\label{individuals-supply-schedule-and-supply-curve}

\subsubsection{Market supply and individual
supplies}\label{market-supply-and-individual-supplies}

\subsubsection{Law of Supply}\label{law-of-supply}

The law of supply states that, the quantity supplied of a good increases
when the price of the good increases, and vice versa, provided all other
factors that affect suppliers' decisions are unchanged.

\subsubsection{Law of Supply:
Explanation}\label{law-of-supply-explanation}

How can we make sense of the numbers in Ben's supply schedule? The best
guess is that his costs must be something like the cost schedule below.

\subsection{Shifts in the Supply Curve: What causes
them?}\label{shifts-in-the-supply-curve-what-causes-them}

How could Ben's supply have increased?

\section{Equilibrium}\label{equilibrium}

Now it is time to say something about how buyers and sellers
collectively determine the market outcome. To do this, we assume
\emph{equilibrium}.

The theory of supply and demand assumes that the market price
automatically reaches a level at which the quantity demanded equals the
quantity supplied.

\subsection{Can we justify the assumption of
equilibrium?}\label{can-we-justify-the-assumption-of-equilibrium}

If the market price exceeds equilibrium price, then the quantity
supplied exceeds the quantity demanded. This is called \emph{excess
supply} or a \emph{surplus}. In such a situation, sellers will be forced
to cut their prices, thereby moving the market price closer to
equilibrium.

If the market price is less than the equilibrium price, then the
quantity supplied is less than the quantity demanded. This is called
\emph{excess demand} or a \emph{shortage}. Sellers will increase the
market price, because too many buyers are chasing too few goods, thereby
moving the market price closer to the equilibrium price.

The market price of a good adjusts to bring the quantity supplied and
the quantity demanded for that good into balance.

\subsubsection{Equilibrium: skepticism
required}\label{equilibrium-skepticism-required}

Although the Law of Supply and Demand is a good place to start the
discussion of prices, it should not be taken to be the gospel truth. In
some cases the price might get stuck at some other level and quantity
supplied and quantity demanded may not be equal.

One example is unemployment. Unemployment is a failure of equilibrium
when the wage is too high and stuck there.

\section{Making predictions}\label{making-predictions}

Is the theory of supply and demand of any use? Let's use the theory to
make some predictions. We have seen some of the factors that shift the
demand and supply curves. Such shifts change the equilibrium outcome.

Using the supply-demand theory we can try to predict the consequences
of:

\begin{itemize}
\tightlist
\item
  alternative policy proposals, and
\item
  events outside our control that we need to be ready for.
\end{itemize}

\subsection{How an Increase in Demand Affects the
Equilibrium}\label{how-an-increase-in-demand-affects-the-equilibrium}

\subsection{How a Decrease in Supply Affects the
Equilibrium}\label{how-a-decrease-in-supply-affects-the-equilibrium}

\subsubsection{Eggflation! Supply Decrease in the
News}\label{eggflation-supply-decrease-in-the-news}

\href{https://youtu.be/WAwEXycDhzY}{What's causing the price of eggs to
skyrocket nationwide}, PBS Newshour, YouTube, January 30, 2023

\subsection{A Shift in Both Supply and
Demand}\label{a-shift-in-both-supply-and-demand}

\subsubsection{Exercises}\label{exercises}

Can you predict:

\begin{itemize}
\tightlist
\item
  The effect of the Covid-19 pandemic:

  \begin{itemize}
  \tightlist
  \item
    \ldots{} on the price of gasoline?
  \item
    \ldots{} on the price of Manhattan real estate?
  \end{itemize}
\item
  The effect of a rise in the price of oil on the market for:

  \begin{itemize}
  \tightlist
  \item
    Hybrid cars
  \item
    Real estate
  \item
    Staple foods (corn, wheat, rice)
  \end{itemize}
\item
  The effect of the development of cheaper and better batteries for
  electric cars on the market for:

  \begin{itemize}
  \tightlist
  \item
    traditional cars
  \item
    gas
  \end{itemize}
\end{itemize}

\section{The Scope of the theory of supply and
Demand}\label{the-scope-of-the-theory-of-supply-and-demand}

The theory can be applied to other kinds of markets, such as:

\begin{itemize}
\tightlist
\item
  Factor/resource markets
\item
  Assets markets
\item
  Prediction markets
\end{itemize}

\bookmarksetup{startatroot}

\chapter{Elasticity}\label{elasticity}

In Chapter~\ref{sec-supply_demand}, we saw that:

\begin{itemize}
\tightlist
\item
  The quantity demanded of a consumer good is affected by its price, the
  prices of other related goods, buyers' incomes, etc., and
\item
  The quantity supplied of a good is affected by its price, the prices
  of raw materials, technology, etc.
\end{itemize}

But these statements are pretty vague.

It would be more specific if we said that the quantity demanded of a
consumer good increases when its price decreases, assuming that all the
other factors that affect the quantity demanded are unchanged.

But even this statement is a bit vague because it does not answer the
``how much'' question: By how much would the quantity demanded of a
consumer good increase when its price decreases by a certain amount.

In this chapter, I will discuss how we could use the concept of
\emph{elasticity} to measure the effect of one variable on another. This
will help us make statements that are \emph{quantitative} rather than
merely \emph{qualitative}.

\section{Definition of Elasticity}\label{definition-of-elasticity}

Elasticity is a measure of the strength of the effect of one variable on
another. In other words, elasticity is a measure of the
\emph{responsiveness} of one variable to another.

Suppose there are two variables, \emph{x} and \emph{y}. Let \emph{x} be
the \emph{cause} and \emph{y} be the \emph{consequence}. The ``\emph{x}
elasticity of \emph{y}'' is a measure of the effect on \emph{y} of
changes in \emph{x}, when all other factors that affect \emph{y} are
unchanged. In other words, the ``\emph{x} elasticity of \emph{y}'' is a
measure of the responsiveness of \emph{y} to changes in \emph{x}, when
all other factors that affect \emph{y} are unchanged.

But even this definition is not precise enough to \emph{measure} the
responsiveness of one variable to another. So, let's tighten up the
definition.

Key definition: The ``\emph{x} elasticity of \emph{y}'' is the percent
increase in \emph{y} when there is a one percent increase in \emph{x},
and all other factors that affect \emph{y} are unchanged.

\[
x \text{ elasticity of } y=\frac{\text{% increase in } y}{\text{% increase in }x}
\]

\subsection{Four Important
Elasticities}\label{four-important-elasticities}

\subsection{Four Important Elasticities: Price Elasticity of
Demand}\label{four-important-elasticities-price-elasticity-of-demand}

Let \emph{x}, the cause, be the \emph{price} of a commodity. Let
\emph{y}, the consequence, be the \emph{quantity demanded} of the
commodity. The \emph{price elasticity of demand} is the percent increase
in the quantity demanded of a commodity when there is a one percent
increase in the price of the commodity, and all other factors that
affect quantity demanded are unchanged.

\[
\text{price elasticity of demand}=\frac{\text{% increase in quantity demanded}}{\text{% increase in price}}
\]

\subsection{Four Important Elasticities: Income Elasticity of
Demand}\label{four-important-elasticities-income-elasticity-of-demand}

Let \emph{x}, the cause, be the \emph{income} of the buyers. Let
\emph{y}, the consequence, be the \emph{quantity demanded} of a
commodity. The \emph{income elasticity of demand} is the percent
increase in the quantity demanded of a commodity when there is a one
percent increase in the buyers' income, and all other factors that
affect quantity demanded are unchanged.

\[
\text{income elasticity of demand}=\frac{\text{% increase in quantity demanded}}{\text{% increase in income}}
\]

\subsection{Four Important Elasticities: Cross Price Elasticity of
Demand}\label{four-important-elasticities-cross-price-elasticity-of-demand}

Let \emph{x}, the cause, be the \emph{price} of a commodity. Let
\emph{y}, the consequence, be the \emph{quantity demanded of a different
commodity}. The \emph{cross-price elasticity of demand} is the percent
increase in the quantity demanded of a commodity when there is a one
percent increase in the price of another commodity, and all other
factors that affect quantity demanded are unchanged.

\[
\text{cross price elasticity of demand}=\frac{\text{% increase in quantity demanded of Good B}}{\text{% increase in the price of Good A}}
\]

\subsection{Four Important Elasticities: Price Elasticity of
Supply}\label{four-important-elasticities-price-elasticity-of-supply}

Let \emph{x}, the cause, be the \emph{price} of a commodity. Let
\emph{y}, the consequence, be the \emph{quantity supplied} of the
commodity. The \emph{price elasticity of supply} is the percent increase
in the quantity supplied of a commodity when there is a one percent
increase in the price of the commodity, and all other factors that
affect quantity supplied are unchanged.

\[
\text{price elasticity of supply}=\frac{\text{% increase in quantity supplied}}{\text{% increase in price}}
\]

\subsection{What we need to know about each of our four
elasticities}\label{what-we-need-to-know-about-each-of-our-four-elasticities}

For each of the four elasticities we have just seen, we need to know:

\begin{itemize}
\tightlist
\item
  how to measure it, and
\item
  what makes it high in some situations and low in other situations, and
\item
  why is it useful
\end{itemize}

\section{Price elasticity of demand
(P.E.D.)}\label{price-elasticity-of-demand-p.e.d.}

The price elasticity of demand is a measure of how strongly the quantity
demanded of a good responds to a change in the price of that good.
Specifically, the price elasticity of demand is the percent increase in
the quantity demanded of a commodity when there is a one percent
increase in the price of the commodity, and all other factors that
affect quantity demanded are unchanged.

\[
\text{Price Elasticity of Demand}=\frac{\text{% Increase in Quantity Demanded}}{\text{% Increase in Price}}
\]

With this definition, we may be able to numerically calculate the P.E.D.
of, say, strawberry ice cream if we had the necessary data. For example,
if on one occasion we observed that the price increased 10 percent and
the quantity purchased decreased 20 percent, then, assuming that all
other factors that affect the quantity demanded/purchased were
unchanged, we can calculate that the price elasticity of demand
strawberry ice cream is -2.

The price increase was +10 percent. As a decrease is expressed as a
negative increase, the increase in quantity demanded was -20 percent.
Therefore, a one percent increase in price caused a -20/+10 = -2 percent
increase in quantity demanded. Therefore, in this case at least, the
P.E.D. of strawberry ice cream is determined to be -2.

\[
\text{Price Elasticity of Demand}=\frac{\text{% Increase in Quantity Demanded}}{\text{% Increase in Price}}=\frac{-20}{+10}=-2
\] Recall that in Chapter~\ref{sec-supply_demand}, we saw examples of
numerical data on price and quantity demanded in demand schedules/tables
and demand curves.

\subsection{We often drop the negative sign in PED
computations}\label{we-often-drop-the-negative-sign-in-ped-computations}

Recall the Law of Demand from Chapter~\ref{sec-supply_demand}. It says
that price and quantity demanded move in \emph{opposite} directions
(provided all the other factors that affect buyers' decisions are
unchanged). Therefore, the percent increase in price and the percent
increase in quantity demanded will always be of opposite signs.
Therefore, the P.E.D. formula in the previous slide will \emph{always be
negative}. For this reason, it is quite common to ignore the sign.

For the rest of this course, I will ignore the negative sign of the PED.
Formally, I will add a negative sign to the PED formula to get rid of
the unnecessary negative sign.

\begin{equation}\phantomsection\label{eq-PED_computed}{
\text{Price Elasticity of Demand}=-\frac{\text{% Increase in Quantity Demanded}}{\text{% Increase in Price}}=-\frac{-20}{+10}=+2
}\end{equation}

\subsection{What are some of the factors that the PED depends
on?}\label{what-are-some-of-the-factors-that-the-ped-depends-on}

Which commodity will have the \emph{lower} price elasticity of demand?

\begin{itemize}
\tightlist
\item
  gasoline or movies?
\item
  insulin injections (for diabetes patients) or music downloads (on
  iTunes)?
\item
  alcoholic beverages in general or Miller beer?
\end{itemize}

A 10 percent increase in gasoline prices reduces gasoline consumption by
about 2.5 percent after a year and about 6 percent after five years.
Why?

Key Idea: P.E.D. of a consumer good tends to be higher:

\begin{itemize}
\tightlist
\item
  if it has many close substitutes
\item
  if it is a luxury commodity

  \begin{itemize}
  \tightlist
  \item
    That is, if the good's income elasticity of demand is high
  \end{itemize}
\item
  if spending on the good is a large portion of total spending
\item
  if it is narrowly (rather than broadly) defined
\item
  if buyers are given more time to adjust to a price change
\end{itemize}

\section{Income Elasticity of Demand
(IED)}\label{income-elasticity-of-demand-ied}

Key Definition: The income elasticity of demand (IED) is the percent
increase in the quantity demanded of a commodity when there is a one
percent increase in the buyers' incomes, and all other factors that
affect quantity demanded are unchanged.

\[
\text{income elasticity of demand}=\frac{\text{% increase in quantity demanded}}{\text{% increase in income}}
\] Example: Suppose income increases 10\%, and quantity demanded
decreases 20\%. Then income elasticity of demand is -20/+10 = -2.

Note that I.E.D. can be positive, negative, or zero.

\subsection{Normal Goods and Inferior
Goods}\label{normal-goods-and-inferior-goods}

Normal Goods have I.E.D. \textgreater{} 0. Inferior Goods have I.E.D.
\textless{} 0.

An increase in income leads to an increase in the quantity demanded for
normal goods, but decreases the quantity demanded for inferior goods.

\subsection{Necessities and Luxuries}\label{necessities-and-luxuries}

When I.E.D. \textless{} 1 for a commodity, it is called \emph{income
inelastic} or a \emph{necessity}.

Examples of necessities include food, fuel, clothing, utilities, and
medical services.

When I.E.D. \textgreater{} 1 for a commodity, it is called \emph{income
elastic} or a \emph{luxury}.

Examples of luxuries include sports cars, furs, and expensive foods.

\subsection{High Income Elasticity Implies High Price
Elasticity}\label{high-income-elasticity-implies-high-price-elasticity}

In the discussion of the Law of Demand in
Chapter~\ref{sec-supply_demand}, we saw that the effect of a change in
the price of a good on its quantity demanded is the sum of:

\begin{itemize}
\tightlist
\item
  The substitution effect, and
\item
  The income effect.
\end{itemize}

The higher the IED, the bigger the income effect. Therefore, the higher
the IED, the higher the PED. Low IED implies low PED. High IED implies
high PED.

\section{Computing percentage
increases}\label{computing-percentage-increases}

Consider the data in the following table:

\begin{longtable}[]{@{}ll@{}}
\caption{Hypothetical Data}\label{tbl-pct_increases}\tabularnewline
\toprule\noalign{}
\% Increase in Price & \% Increase in Quantity Demanded \\
\midrule\noalign{}
\endfirsthead
\toprule\noalign{}
\% Increase in Price & \% Increase in Quantity Demanded \\
\midrule\noalign{}
\endhead
\bottomrule\noalign{}
\endlastfoot
10 & -20 \\
\end{longtable}

The price elasticity of demand is easy to calculate in this case, as
shown in Equation~\ref{eq-PED_computed}.

Now consider the slightly different case of two rows of a demand
schedule:

But how do we calculate the PED in this case? Now, we'll need to
calculate the percent increases from the data And then plug them into
the PED formula Price Elasticity of Demand: Calculation Step 1: We
arbitrarily call one of the rows Row A and the other row Row B. Step 2:
The percentage increase in price is: ``Increase'' /``Average'' ×100
(𝑃\_𝐵−𝑃\_𝐴)/((𝑃\_𝐵+𝑃\_𝐴)/2)×100

(4−2)/((4+2)/2)×100=2/3×100=𝟔𝟔.𝟕

Price Elasticity of Demand: Calculation Step 1: We arbitrarily call one
of the rows Row A and the other row Row B. Step 3: The percentage
increase in quantity demanded is: ``Increase'' /``Average'' ×100
(〖𝑄𝐷〗\_𝐵−〖𝑄𝐷〗\_𝐴)/((〖𝑄𝐷〗\_𝐵+〖𝑄𝐷〗\_𝐴)/2)×100

(8−10)/((8+10)/2)×100=(−2)/9×100=−𝟐𝟐.𝟐

Price Elasticity of Demand: Calculation The price elasticity of demand
is now easy to calculate ``PED''=−``\% increase in Quantity Demanded''
/``\% increase in Price'' =−(−22.2)/66.7=0.33 So, a 66.6 percent
increase in price caused a 22.2 percent decrease in the quantity
demanded Therefore, a 1 percent increase in price caused a 0.33 percent
decrease in the quantity demanded That's the PED

\subsection{Price Elasticity of Demand: Midpoint
Formula}\label{price-elasticity-of-demand-midpoint-formula}

The formula used here is called the midpoint formula. Note that you'll
get the same result if you switched the row names. Try it! Price
Elasticity of Demand: Example Calculate the PED using the two points
highlighted on this demand curve Step 1: Turn the prices and quantities
for the two highlighted points into a demand schedule Step 2: Then
calculate PED as in the previous example Price Elasticity of Demand:
Example The percentage increase in price is:
(𝑃\_𝐵−𝑃\_𝐴)/((𝑃\_𝐵+𝑃\_𝐴)/2)×100=(4−5)/((4+5)/2)×100=−22.2 The percentage
increase in quantity demanded is:
(〖𝑄𝐷〗\_𝐵−〖𝑄𝐷〗\_𝐴)/((〖𝑄𝐷〗\_𝐵+〖𝑄𝐷〗\_𝐴)/2)×100=(100−50)/((100+50)/2)×100=66.7
``PED''=−``\% increase in Quantity Demanded'' /``\% increase in Price''
=−66.7/(−22.2)=3

\subsection{PED and the shape of the demand
curve}\label{ped-and-the-shape-of-the-demand-curve}

Elastic, Unit-Elastic and Inelastic Demand Demand can be Inelastic
Unit-elastic Elastic \ldots{} depending on the magnitude of the P.E.D.
Inelastic Demand: P.E.D. \textless{} 1 Suppose: Price increases by 10\%
Quantity demanded decreases by 5\%. \% change in quantity is smaller
than the \% change in price Here P.E.D. = -(-5/10) = 0.5 \textless{} 1
Recall that I am ignoring the negative sign of the P.E.D. In this case,
we say demand is inelastic Unit-Elastic Demand: P.E.D. = 1 Suppose:
Price increases by 10\% Quantity demanded decreases by 10\%. \% change
in quantity is equal to the \% change in price Here P.E.D. = -(-10/10) =
1 Recall that I am ignoring the negative sign of the P.E.D. In this
case, we say demand is unit-inelastic Elastic Demand: P.E.D.
\textgreater{} 1 Suppose: Price increases by 10\% Quantity demanded
decreases by 30\%. \% change in quantity is greater than the \% change
in price Here P.E.D. = -(-30/10) = 3 \textgreater{} 1 Recall that I am
ignoring the negative sign of the P.E.D. In this case, we say demand is
elastic Perfectly Inelastic and Perfectly Elastic Demand Key Definition:
Perfectly Inelastic Demand Quantity demanded does not respond at all to
price changes. P.E.D. = 0. Key Definition: Perfectly Elastic Demand
Quantity demanded changes infinitely with any change in price. P.E.D. =
infinity. The Variety of Demand Curves Because P.E.D. measures how
strongly quantity demanded responds to the price, it is closely related
to the slope of the demand curve. Key Idea: The higher the price
elasticity of demand, the flatter the demand curve.

\subsection{PED affects the link between price and sales
revenue}\label{ped-affects-the-link-between-price-and-sales-revenue}

Total revenue (TR) is the amount received by sellers from the sale of a
good. It is usually computed as the price of the good (P) times the
quantity sold (Q): \(TR=P\times Q\).

\subsubsection{Total Revenue,
Graphically}\label{total-revenue-graphically}

Elasticity and Total Revenue: Inelastic Demand Suppose demand is
inelastic Specifically, suppose Price increases by 10\%, and Quantity
demanded decreases by 4\%. Elasticity and Total Revenue: Elastic Demand
Suppose demand is elastic Specifically, suppose Price increases by 10\%,
and Quantity demanded decreases by 25\%. Elasticity and Total Revenue:
Unit-elastic Demand Suppose demand is unit-elastic Specifically, suppose
Price increases by 10\%, and Quantity demanded decreases by 10\%.

\subsubsection{Elasticity of a Linear Demand
Curve}\label{elasticity-of-a-linear-demand-curve}

Note that demand can change from elastic to unit-elastic to inelastic as
the price changes. That is, in addition to the other factors discussed
before, PED also depends on the price.

\section{Other important
elasticities}\label{other-important-elasticities}

\subsection{Cross Price Elasticity of Demand
(CPED)}\label{cross-price-elasticity-of-demand-cped}

Key Definition: The cross price elasticity of demand (C.P.E.D.) measures
the responsiveness of the quantity demanded of one good to changes in
the price of some other good. ``CPED''=``\% increase in the quantity
demanded of Good X'' /``\% increase in the price of Good Y'' Suppose:
The price of Coke increases 2\%, and The consumption of Pepsi increases
20\%. The C.P.E.D. for Coke with respect to Pepsi is +20/+2 = +10.
Substitutes: C.P.E.D. \textgreater{} 0 Key Definition: When the C.P.E.D.
for one good with respect to another is positive, the two goods are
called substitutes. In the last slide's example, the C.P.E.D. for Coke
with respect to Pepsi was +10, which is positive. This confirms our
common-sense intuition that Coke and Pepsi are substitutes. Complements:
C.P.E.D. \textless{} 0 Key Definition: When the C.P.E.D. for one good
with respect to another is negative, the two goods are called
complements. Suppose: the price of gasoline increases 50\% the sale of
cars decreases 10\% The C.P.E.D. is -10/+50 = -0.2, which is negative.
This is what one would expect, given our common-sense notion that gas
and cars are complements.

\subsection{Price Elasticity of Supply
(PES)}\label{price-elasticity-of-supply-pes}

Price elasticity of supply is a measure of how strongly the quantity
supplied of a good responds to a change in the price of that good. More
specifically, the price elasticity of supply is the percent increase in
the quantity supplied of a commodity when there is a one percent
increase in the price of the commodity, and all other factors that
affect quantity supplied are unchanged.

\subsubsection{Determinants of Elasticity of
Supply}\label{determinants-of-elasticity-of-supply}

What makes the price elasticity of supply high in some cases and low in
others?

The price elasticity of supply is high if the production technology
enables sellers to change production easily.

Beach-front land has inelastic supply. Books, cars, or manufactured
goods have elastic supply. if suppliers have time to respond to a price
change So, supply is more elastic in the long run.

\section{APPLICATIONS OF SUPPLY, DEMAND, AND
ELASTICITY}\label{applications-of-supply-demand-and-elasticity}

\subsection{Can good news for farming be bad news for
farmers?}\label{can-good-news-for-farming-be-bad-news-for-farmers}

What happens to wheat farmers and the market for wheat when university
agronomists discover a new wheat hybrid that is more productive than
existing varieties?

\subsection{How does the effect of coordinated cuts to crude oil
production affect oil
prices?}\label{how-does-the-effect-of-coordinated-cuts-to-crude-oil-production-affect-oil-prices}

\subsection{How does the debate over drugs legalization depend on the
PED for
drugs?}\label{how-does-the-debate-over-drugs-legalization-depend-on-the-ped-for-drugs}

\subsection{Housing in the Bay Area}\label{housing-in-the-bay-area}

\section{Video}\label{video}

\begin{itemize}
\tightlist
\item
  https://youtu.be/0Flsg\_mzG-M
\item
  https://youtu.be/AcVaTIW-DGQ
\end{itemize}

\bookmarksetup{startatroot}

\chapter{Data}\label{data}

\section{Description}\label{description}

\section{Missing value analysis}\label{missing-value-analysis}

\bookmarksetup{startatroot}

\chapter{Results}\label{results}

\bookmarksetup{startatroot}

\chapter{Interactive graph}\label{interactive-graph}

\phantomsection\label{plot}

\bookmarksetup{startatroot}

\chapter{Conclusion}\label{conclusion}


\backmatter

\printindex

\end{document}
